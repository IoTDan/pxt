\section{The BBC and the Foundation}
\label{sec:mef}

The BBC's history with computing and education goes back to the early 1980's
and the BBC Computer Literacy Project, which featured the BBC Micro, 
a 6502-based microcomputer designed and produced by Acorn Computers Ltd. (referred
to at times as the ``British Apple'').  The project was very sucessful:
more than 80\% of UK classrooms had a BBC Micro and many of today's 
computing professionals from the UK first encountered computing through
the BBC Micro.

In December 2014, the BBC issued an Request for Participation
for ``Delivery of a hands-on learning experience for the Make it Digital season'',
which was the micro:bit project.
Twenty-nine partners were invited to contribute hardware, software, services, 
teaching materials, packing/distribution, logistics, events and funding.
Work on the project commenced in February 2015, with delivery of
a web site/app in September 2015 (which was critical
for training teachers) and delivery of the micro:bits in the second
half of the 2015-2016 school year.

% from: http://microbit.org/about/

Founded in September 2016,
the Micro:bit Educational Foundation is a non-profit organization
legally established with the support of its founding partners~\footnote{ARM,
Amazon, BBC, British Council, IET, Lancaster University, Microsoft,
Nominet, and Samsung}. 
The Foundation's Mission Statement is to: 
\begin{itemize}
\item  enable and inspire all children to participate in the digital world, 
with particular focus on girls and those from disadvantaged groups.
\item make micro:bit the easiest and most effective learning tool for digital skills and creativity.
\item work in collaboration with educators to create and curate exceptional 
curriculum materials, training programmes and resources.
\item build and support communities of educators and partners 
to remove the barriers to learning digital skills
\end{itemize}
The Foundation works to make micro:bits available for purchase (singly and in bulk)
around the world through resellers.~\footnote{Currently in 
Australia, Belgium, Brazil, Canada, China, Croatia, Czech Republic, 
Denmark, Estonia, Finland, France, Germany, Hong Kong, Hungary, India, 
Ireland, Israel, Italy, Japan, Latvia, Lithuania, Luxembourg, Malaysia, 
Netherlands, New Zealand, Norway, Poland, Singapore, Slovak Republic, 
South Africa, Spain, Sweden, Switzerland, Taiwan, Thailand, UK, and the US.}
The Foundation redistributes the bulk of any surplus money 
generated into providing free devices to exceptional 
micro:bit educational programmes across the globe.

% country deployments