\section{Context}
\label{sec:domain}

\subsection{Physical Computing}
Physical computing: computing interacting with our physical environment (as opposed to just
living on a screen, like a computer game); ``cyberphysical systems'', ``embedded (reactive) systems''.
Physical computing lives in the spaces between computing and many other disciplines:
art, industrial design, health, environmental monitoring, etc.

% Finally, it is worth noting that the benefits of physical computing aren’t limited to CS education. 
% There are diverse connections to other STEM subjects, such as the simulation of behaviour in biology, 
% the collection and analysis of measurements in physics and logical mathematical operations [7]. 
% Physical computing also connects into the arts and humanities, with application to topics ranging 
% from interactive art pieces to geography and dance [5], [7]. 

Physical computing benefits:
\begin{itemize}
\item broad reach because of diverse applications of physical computing;
\item increased motivation and connections because of tangible visible outcome (rather than virtual on screen);
\item leverage fine arts, music, design, etc. in projects;
\item learning by doing: many ways to achieve goal (no single correct solution)
\item natural division of labor for more complex projects (design, hardware, software, ...)
\item full system view of computing: hardware and software working together.
\end{itemize}

% from Steve and Sue's white paper:

% We summarize the benefits of physical computing in the classroom thus:
%
% Motivation:
%
% Increased motivation for students, including those from diverse backgrounds, 
% because the learning experience and the outcome are visible not virtual. 
% This is especially true when a programming task delivers a practical, meaningful device.~\cite{XYZ}
%
% Tangibility & Interactivity:	
%
% The tangible nature of physical devices helps students make natural connections. 
% Iteratively debugging and refining tangible systems helps them better understand
% programming concepts and the software development process.~\cite{XYZ}
%
% Creativity:
%
% Students naturally relate to the physical nature of the task, unleashing creativity 
% in terms of what they build and thereby strengthening engagement with the task.~\cite{XYZ}
%
% Learning by doing:
%
% Physical computing projects promote trial-and-error because there are many ways to 
% achieve most goals rather than a single correct solution. This supports learning by 
% doing in an iterative fashion. 
%
% Collaboration:
% 
% Working with devices lends itself to group work – different roles include enclosure
% design, hardware interfacing, algorithm design and user interaction. Groups of students
% can readily cooperate (or compete!) because of the physical nature of challenges and tasks.
%
% Holistic View of Computing Education:
%
% Computer systems are comprised of hardware as well as software, and computer science is not
% just about programming. It is important for students to learn about the physical hardware 
% components of computer systems and how they work, especially given the emergence of the 
% internet of things (IoT).
%
% Engages the Whole Learner:
%
% The physical nature of the work engages the whole student – both their mind and their body,
% making the learning process a deep, immersive experience.




\subsection{Wiring, Arduino and the BBC micro:bit}

% from thesis of Hernando Barragán:
% - http://people.interactionivrea.org/h.barragan/thesis/thesis_low_res.pdf 
% - Wiring: Prototyping Physical Interaction Design
% - June 2004

% http://wiring.org.co/

% https://globenewswire.com/news-release/2017/05/19/988294/0/en/Arduino-Welcomes-Hernando-Barrag%C3%A1n-as-Arduino-Chief-Design-Architect.html

To help explain the BBC micro:bit design, it's very instructive to understand
Hernando Barragan's 2003 Master's thesis, ``Wiring: Prototyping Physical Interaction Design'',
the inspiration for the Arduino system~\cite{Barragan}. His objective was to make it easier
for non-technical creators, such as artists and designers, to leverage
electronics in their their work by simplifying the hardware and programming
experience. In particular, he said of existing work:
``Current prototyping tools for electronics and programming are mostly targeted 
to engineering, robotics and technical audiences.''  
Of Wiring's design, he identified the following key concepts:
\begin{itemize}
\item a simple cross-platform integrated development environment (IDE) to create so-called ``sketches'';
\item a simplified APIs to access the microcontroller's resources;
\item leverage of open source compiler/linker toolchain, transparent to the end user;
\item a bootloader to make it easy to upload a compiled sketch to the microcontroller;
\end{itemize}
Also make Wiring (hardware and software) open source.

But, still some issues:
\begin{itemize}
    \item reliance on the C language and C compiler (needs to be installed)
    \item very poor experience in IDE
    \item USB bootloader requires device drivers on some systems
\end{itemize}

BBC micro:bit inherits the raw PCB nature of Arduino (everything is visible to the end user).

First key idea of the BBC micro:bit: NO WIRING REQUIRED!


