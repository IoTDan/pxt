% ISTE 2018:

% The Coding Continuum: Creating with Micro:bit
% https://conference.iste.org/2018/program/search/detail_session.php?id=110816010
% Hands-on Making and Coding with Micro:bit
% https://conference.iste.org/2018/program/search/detail_session.php?id=110832835
% Blink Blink = Good Health: Young Girls Design a Wearable With Micro:bit
% https://conference.iste.org/2018/program/search/detail_session.php?id=110759398


% Topics covered must reach out to a very broad technical audience.
% Articles in Communications should be aimed at the broad computing and information technology community.

% A Contributed Article should set the background and provide introductory references, 

% define fundamental concepts, 

% compare alternate approaches, 

% Arduino
% Scratch (tethered)
% MicroPython

% and explain the significance or application of a particular technology 

% or result by means of well-reasoned text and pertinent graphical material. 

% The use of sidebars to illustrate significant points is encouraged.

% Full-length Contributed Articles should consist of up to 4,000 words, 
% contain no more than 25 references, 3-4 tables, 3-4 figures, 
% and be submitted to: http://cacm.acm.org/submissions.

% Submissions to the Contributed Articles section should be accompanied by a cover letter indicating:
% • Title and the central theme of the article; 
% • Statement addressing why the material is important to the computing field and of value to the Communications reader;
% • Names and email addresses of three or more recognized experts who would be considered appropriate to review the submission.


% from the BBC RFP

% Make it Digital is a BBC initiative to inspire a new generation to get creative with 
% coding, programming and digital technology. With a major focus in 2015, Make it Digital 
% will help all audiences see how Britain has helped shape the digital world, why digital 
% skills matter and their growing importance to our future. For younger audiences, Make it 
% Digital will help them discover the world of digital, see their creative potential in it 
% and inspire them to take their first steps in computational thinking and digital skills. 

% In September/October [2015], as part of the overall Make It Digital project, BBC Learning aims 
% to give away one million small, programmable, wearable devices. The device can be used 
% in classrooms and the home to create an engaging hands-­‐on learning experience that 
% allows any level of coder from absolute beginner to advanced maker to get involved and 
% be part of something exciting. Approximately 800,000 devices will go to every child in 
% the UK at Year 7 (tbc) and a further 200,000 will be made available to distribute 
% through competitions and other activities targeting other groups covering both children 
% and adults. 

% We anticipate that an online coding site will support the devices. The site will allow 
% owners to create programmes to run on their own devices and/or share programmes and code 
% with others. 
